\section{Bonds and Interest Rates} 
In this section, we will introduce the fixed income market. The simplest example is the zero-coupon bond; if it pays 1\$ at maturity time $T$, we want to decide its value at time $0$. Assume the interest rate $r$ is fixed. Then the price of the zero-coupon bond at time $t$ is given by
$$B(t, T) = e^{ -r(T-t) }.$$
Or it can be written as
\begin{align*}
	\dd B_t = r B_t \dd t, \\
	B(T,T) = 1.
\end{align*}
Moreover, we are also interested in the relation between the interest rate and the maturity time $T$. Based on the model above, we can solve $r$ for 
$$r(T) = - \frac{1}{T- t}\log B(t,T) \equiv r;$$
it is called the \textbf{yield curve}. 


\subsection*{EXAMPLES}

\begin{example}[Vasicek model]
	Assume the rate $r$ for a bond with maturity time $T$ is given by
	$$\dd r_t = a(m - r_t)\dd t + \sigma \dd W_t$$
	where $a$, $m$, and $\sigma$ are constant. Its price at time $t$ is
	$$B(t,T) = \EE^\QQ \left(  e^{-\int^T_t r_s \dd s} \bst \FF_t  \right)$$
	\begin{itemize}
		\item In this example, we want to find the precise representation of $B(0,T)$. First, we notice that $\int^T_0 r_s \dd s$ is a normal distribution. Then we can compute its mean and variance:
		\begin{align*}
			\EE^\QQ \left[ \int^T_0 r_s \dd s  \right] =    \int^T_0  \EE^\QQ(r_s) \dd s  = 0 
		\end{align*}
		by Fubini theorem; and 
		\begin{align*}
			\EE^\QQ \left[ (\int^T_0 r_s \dd s )^2 \right] &= \int^T_0 \int^T_0 \Cov( r_s, r_u ) \dd s \dd u \\
			&= \int^T_0 \int^T_0  \frac{\sigma^2}{2a}\left( e^{-a(u-s)} - e^{-a(u+s)} \right) \dd s \dd u \\
			&= 
		\end{align*}
		
		Second, by the moment generating function of Gaussian random variable, 
		\begin{align*}
			B(0, T) = 
		\end{align*}
		
		\item Then we can compute its yield curve.
		\begin{align*}
			Y(0, T) &= -\frac{1}{T - t} \log B(0, T) \\
			&=
		\end{align*}
		
		\item Comment on this curve
	\end{itemize}
\end{example}

\begin{example}[CIR model]
	Now assume the rate $r$ is defined as 
	$$\dd r_t = a(m - r_t) \dd t + \sigma \sqrt{r_t} \dd W_t.$$
	Then its price at time $0$ is also
	$$P(0, T) = \EE^\QQ \Big[ \int^T_t r_s \dd s  \Big].$$
	However, because in this case $r_t$ is chi-square distributed, we cannot write a nice formula for its price.
\end{example}

\subsection*{FORWARD MEASURE}
\begin{quotation}
	``In finance, a T-forward measure is a pricing measure absolutely continuous with respect to a risk-neutral measure but rather than using the money market as numeraire, it uses a bond with maturity T. The use of the forward measure was pioneered by Farshid Jamshidian (1987), and later used as a means of calculating the price of options on bonds.''
	
	\attrib{\hyperlink{https://en.wikipedia.org/wiki/Forward_measure}{Forward measure - Wikipedia}}
\end{quotation}

\begin{lem}
	Let $(X,Y)$ be a bi-normal distribution under $\PP^\star$; that is,
	$$(X,Y) \sim_{\PP^\star} \mathrm{Normal}\big( \begin{pmatrix}
	\mu_X \\
	\mu_Y
	\end{pmatrix} , \begin{pmatrix}
	\sigma^2_X & \rho \sigma_X \sigma_Y \\
	\rho \sigma_X \sigma_Y & \sigma_Y^2
	\end{pmatrix} \big).$$ 
	And assume $\PP \ll \PP^\star$ with density
	$$\frac{\dd \PP}{\dd \PP^\star} = \frac{e^{-\lambda X}}{\EE^{\PP^\star} e^{-\lambda X} }.$$
	Then the distribution of $Y$ under $\PP$ is
	$$Y \sim_\PP \mathrm{Normal}\left(  \mu_Y -\lambda \rho\sigma_X \sigma_Y,  \sigma_Y^2  \right).$$
\end{lem}
\begin{proof} 
	For convenience, we use $\EE^\star$ to represent $\EE^{\PP^\star}$. Directly compute the distribution function for $Y$ under $\PP$ as follow:
	\begin{align*}
	\PP( Y\leq t ) &=  \int_\Omega \IndOne_{ \{ Y\leq t \} } \ \dd \PP  = \int_\Omega \IndOne_{ \{ Y\leq t \} }  \frac{\dd \PP}{\ \dd \PP^\ast} \ \dd \PP^\star \\ 
	&= \EE^\star \left(  \IndOne_{ \{ Y\leq t \} } \cdot   e^{-\lambda X}     \right)/\EE^\star( e^{-\lambda X} ) 
	\end{align*} 
	Obviously, $\EE^\star( e^{-\lambda X} ) $ doesn't rely on $t$. So it remains to compute $\EE^\star \left(  \IndOne_{ \{ Y\leq t \} } \cdot   e^{-\lambda X}  \right)$.
 
		 It is well-known that
		$$X| Y \sim_\star N\left(  \mu_X + \rho\frac{ \sigma_X  }{\sigma_Y}(Y-\mu_Y), \sigma_X^2 - {\rho^2 \sigma_X^2  }    \right);$$
		and it implies that 
		\begin{align*}
		\EE^\star(e^{-\lambda X }\bst Y) &= \exp \left\{ - \big[ \mu_X + \rho\frac{ \sigma_X  }{\sigma_Y}(Y-\mu_Y) \big] \lambda + \sigma_X^2 \left[  1 - \rho^2      \right] \frac{\lambda^2}{2} \right\}.
		\end{align*}
		
	  Now we have
		\begin{align*}
		\EE^\star \left(  \IndOne_{ \{ Y\leq t \} } \cdot   e^{-\lambda X}     \right) &= \EE^\star\left[ \IndOne_{ \{ Y\leq t \} } \EE^\star\left( e^{-{\lambda X} }|Y \right) \right]  \\
		&= \EE^\star\left[  \IndOne_{ \{ Y\leq t \} }  \exp \left\{ - \big[ \mu_X + \rho\frac{ \sigma_X  }{\sigma_Y}(Y-\mu_Y) \big] \lambda + \sigma_X^2 \left[  1 - \rho^2      \right] \frac{\lambda^2}{2} \right\} \right] \\
		&= \int^t_{-\infty} \left\{ \exp \left\{ - \big[ \mu_X + \rho\frac{ \sigma_X  }{\sigma_Y}(y -\mu_Y) \big] \lambda + \sigma_X^2 \left[  1 - \rho^2      \right] \frac{\lambda^2}{2} \right\}  \cdot \frac{1}{\sqrt{ 2\pi \sigma_Y^2 }} e^{  - \frac{ (y - \mu_Y)^2 }{  2\sigma_Y^2 }  }  \right\}  \dd y
		\end{align*}
		
		 Take derivative on both sides. The PDF of $Y$ under $\PP$ is 
		\begin{align*}
			f_Y(t) &\propto \exp \left\{ -  \rho \lambda \frac{ \sigma_X  }{\sigma_Y} t   \right\}  \cdot   \exp\{  - \frac{ (t - \mu_Y)^2 }{  2\sigma_Y^2 }  \} \\
			&\propto \exp\{  \frac{(t + \lambda \rho \sigma_X \sigma_Y - \mu_Y )^2}{2\sigma_Y^2}  \}
		\end{align*}
  
	Therefore, the distribution of $Y$ under $\PP$ is 
	$$Y \sim_\PP \mathrm{Normal}(  \mu_Y -\lambda \rho\sigma_X \sigma_Y,  \sigma_Y^2 ).$$
\end{proof}

\begin{example}[Call option on bonds] Assume a Vasicek model for the short-rate process $r$ under the pricing measure $\PP^\star$. We would like to price, at time $0$, a call option with
	maturity $T$ and strike $K$ on a zero-coupon bond with maturity~$T_1 > T$.  
		\begin{itemize}
			\item Let its price at time $t$ with maturity $T$ is $C(t,T)$. Then we directly price the call option at time~$0$:
			\begin{align*}
			C(0,T) &= \EE^\star \left[  e^{ -\int^T_0 r_s \dd s } ( P(T,T_1) - K  )^+  \right] \\
			&= \EE^\star \left[  e^{ -\int^T_0 r_s \dd s } ( P(T,T_1)  - K  )\cdot \IndOne\{ P(T,T_1) \geq K \} \right]  \\
			&= \EE^\star \left[  e^{ -\int^T_0 r_s \dd s } \cdot P(T,T_1)  \cdot \IndOne\{ P(T,T_1) \geq K \} \right] - K\cdot \EE^\star \left[  e^{ -\int^T_0 r_s \dd s } \cdot \IndOne\{ P(T,T_1) \geq K \} \right]
			\end{align*}
			For convenience we denote 
			\begin{align*}
			(F1) &= \EE^\star \left[  e^{ -\int^T_0 r_s \dd s } \cdot P(T,T_1)  \cdot \IndOne\{ P(T,T_1) \geq K \} \right],   \\
			(F2) &= \EE^\star \left[  e^{ -\int^T_0 r_s \dd s } \cdot \IndOne\{ P(T,T_1) \geq K \} \right];
			\end{align*}
			then
			$$C(0,T) = (F1) - K \cdot (F2).$$ 
			\item \textbf{Compute (F1).}
			\begin{align*}
			(F1) &= \EE^\star \left[  e^{ -\int^T_0 r_s \dd s } \cdot P(T,T_1)  \cdot \IndOne\{ P(T,T_1) \geq K \} \right]   \\
			&= P(0,T_1) \cdot \EE^\star \left[  \frac{e^{ -\int^T_0 r_s \dd s }}{P(0,T_1)} \cdot P(T,T_1)  \cdot \IndOne\{ P(T,T_1) \geq K \} \right]   \\
			\end{align*}
			Let $$\frac{ \dd P^{T_1} }{ \dd P^\star } =  \frac{e^{ -\int^T_0 r_s \dd s }}{P(0,T_1)} \cdot P(T,T_1) .$$
			
			So (F1) can be written as
			\begin{align*}
			(F1) = P(0,T_1) \cdot \PP^{T_1}   \{ P(T,T_1) \geq K   \}  . 
			\end{align*}
			
			It remains to compute $\PP^{T_1}   \{ P(T,T_1) \geq K   \} $. 
			
			\item \textbf{Compute (F2).} Similarly, let $$\frac{ \dd P^T }{ \dd P^\star } = \frac{e^{ -\int^T_0 r_s \dd s }}{P(0,T)};$$
			then we find
			\begin{align*}
			(F2) &= \EE^\star \left[  e^{ -\int^T_0 r_s \dd s } \cdot \IndOne\{ P(T,T_1) \geq K \} \right] \\
			&= P(0,T)\cdot \EE^\star \left[  \frac{e^{ -\int^T_0 r_s \dd s }}{P(0,T)} \cdot \IndOne\{ P(T,T_1) \geq K \} \right] \\
			&= P(0,T) \cdot \PP^T \{ P(T,T_1) \geq K \}  
			\end{align*}
			
			\item Now we find
			\begin{align*}
			C(0,T) &= (F1) - K \cdot (F2) \\
			&= P(0,T_1) p_1 - KP(0,T) p_2
			\end{align*}
			where $p_1$ and $p_2$ are given in (3) and (4), respectively. 
		\end{itemize}
			   
\end{example}

\subsection*{DEFAULTABLE BONDS} 
Assume the interest rate is constant. We consider the following approaches to price the zero-coupon bond at time $0$. (\textbf{Note}: when $r$ is not a constant, we change $(T-t)r$ to $\int^T_t r_s \dd s$; and in this case, we cannot take it out from the expectation.)
\paragraph{Merton’s Approach} In this approach, the basic setting is 
\begin{center}
	DEFAULT = $\{ S_T \leq D \}$.
\end{center}
In this setting, the price of bonds is
\begin{align*}
	P^D(t,T) &= \EE^\star_{t,s} \left[ e^{-r(T-t)} \IndOne_{\{ S_T > D \}} \right] \\
	&= e^{-r(T-t)} \PP^\star_{t,s}\left[ S_T > D \right]\\
	&= e^{-r(T-t)} N(d_2)
\end{align*}
where 
$$d_2 = \frac{ \log(\frac{x}{D}) + (r - \frac{\sigma^2}{2})(T-t) }{\sigma\sqrt{T-t}}$$

\paragraph{Black-Cox Approach} Now we consider the first default time $\tau$
\begin{center}
	DEFAULT = $\{ \min_{0\leq t \leq T} S_t \leq D \}$
\end{center}
In this setting, the price of bonds at time $0$ is 
\begin{align*}
	P^D ( 0 , T ) = e ^ { - r T } \left( N \left( d _ { 2 } ^ { + } \right) - \left( \frac { x } { B } \right) ^ { 1 - k } N \left( d _ { 2 } ^ { - } \right) \right)
\end{align*}
where $k =2r/\sigma^2$ and 
$$d _ { 2 } ^ { \pm } = \frac { \pm \log \left( \frac { x } { B } \right) + \left( r - \frac { \sigma ^ { 2 } } { 2 } \right) T } { \sigma \sqrt { T } }.$$

\paragraph{Intensity Based Approach}
In this setting, we are given the intensity of $\IndOne_{\{ \tau > T \}}$. Then the bond price is
\begin{align*}
	P^D (0, T) = e^{-rT} \EE^\star\left[  e^{-\int^T_0 \lambda_s \dd s} \right]
\end{align*}

\subsection*{FORWARD RATE AND YIELD CURVE}
In this subsection, we summarize the previous results together.  
\paragraph{Forward Rate}
$$f(t, T)=\frac{E^{\star}\left\{r_{T} e^{-\int_{t}^{T} r_{s} d s} | r_{t}\right\}}{E^{\star}\left\{e^{-\int_{t}^{T} r_{s} d s} | r_{t}\right\}}$$

\paragraph{Yield Curve} 
\begin{align*}
P(0,T) &= e^{ -Y(0,T) T }\\
Y(0,T) &= -\frac{1}{T} \log P(0,T)
\end{align*}

\paragraph{Yield Spread} 
\begin{align*}
P^D(0,T) &= e^{ -(Y(0,T)+ Y^S(0,T) T }\\
Y^S(0,T) &= -\frac{1}{T} \log \frac{P^D(0,T)}{P(0,T)}
\end{align*}

