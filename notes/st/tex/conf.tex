\section{Confidence Estimation}
\subsection{Confident bounds and confident intervals}
\begin{mydef}Begin with a family $\FF_\Theta$, $\Theta\subset \RR$. 
	\begin{itemize}
		\item For $\alpha \in (0,1)$, $\underline{\theta}(X)$ is a lower confident bound (LCB) for $\theta$ of level $1-\alpha$ if 
		$$\inf_\theta \PP_\theta (\underline{\theta}(X) \leq \theta) \geq 1-\alpha .$$
		
		\item For $\alpha \in (0,1)$, $\bar{\theta}(X)$ is a upper confident bound (UCB) for $\theta$ of level $1-\alpha$ if 
		$$\inf_\theta \PP_\theta (\bar{\theta}(X) \geq \theta) \geq 1-\alpha .$$
		
		\item $(\underline{\theta}(X),\bar{\theta}(X)$ is a level $1-\alpha$ confident interval (CI) if 
		$$\inf_\theta \PP_\theta (\underline{\theta}(x) \leq \theta \leq \bar{\theta}(x))  \geq 1-\alpha.$$
	\end{itemize}
\end{mydef}
\begin{remark}
	Confident bounds and intervals are not unique.
\end{remark}

\begin{exap}
	$X \sim N(\theta, \sigma^2)$. $\sigma$ known. (So $\frac{X-\theta}{\sigma} \sim N(0,1)$.)
	
	We show: A LCB is $\underline{\theta}(X) = X - \sigma z_{1-\alpha}$. Since
	$$\PP_\theta( X-\sigma z_{1-\alpha} \leq \theta ) = \PP( \frac{X-\theta}{\sigma} \leq z_{1-\alpha} ) = 1- \alpha.$$
	
	Similarly, a UCB is $\bar{\theta}(X) = X + \sigma z_{1-\alpha}$. Since
	$$\PP_\theta( X+\sigma z_{1-\alpha} \geq \theta ) = \PP( \frac{X-\theta}{-\sigma} \leq z_{1-\alpha} ) = 1- \alpha.$$
	
	And a CI is $(X-\sigma z_{1-\frac{\alpha}{2}}, X+\sigma z_{1-\frac{\alpha}{2}})$.
\end{exap}

\subsection{Confident sets and uniformly most accuracy (UMA)}
\begin{mydef}\textbf{ }
	\begin{itemize}
		\item Suppose $\underline{\theta}^1,\underline{\theta}^2$ are level $1-\alpha$ lower confident bounds. We say $\underline{\theta}^1$ is \uline{more accurate} than $\underline{\theta}^2$ if for any $\theta \in \Theta$ and $\tilde{\theta}<\theta$,
		$$\PP_\theta( \underline{\theta}^1(X) \leq \tilde{\theta} ) \leq \PP_\theta( \underline{\theta}^2(X) \leq \tilde{\theta} ).$$
		%向下偏离真实参数的概率总是更小
		
		\item
		Let $\underline{\theta}^\ast$ be a level $1-\alpha$ LCB. If for any other level $1-\alpha$ LCB $\underline{\theta}$, $\underline{\theta}^\ast$ is more accurate than $\underline{\theta}$, then $\underline{\theta}^\ast$ is \uline{uniformly most accurate} (UMA). 
	\end{itemize}
\end{mydef}
\begin{remark}
	We try to minimize the false converage rate $\PP_\theta( \underline{\theta}(X)\leq \tilde{\theta} )$. The related notions for UCB are similar.
\end{remark}


\begin{mydef}\textbf{ }
	\begin{itemize}
		\item 
		A set-valued statistic $S: \underline{X} \to 2^\Theta$ is a \uline{level $1-\alpha$ confident set} if
		$$ \inf_\theta \PP_\theta( S(X) \ni \theta ) \geq 1- \alpha.$$
		
		\item
		$S^\ast$ is said to be \uline{uniformly most accurate} if $\forall \theta \in \Theta$, $\tilde{\theta}\neq \theta$, and $S$ another level $1-\alpha$ confident set
		$$\PP_\theta (  S^\ast(X) \ni \tilde{\theta}  ) \leq \PP_\theta(  S(X) \ni \tilde{\theta}  ).$$
	\end{itemize}
\end{mydef}
\begin{comment}
\begin{exap}[No UMA confident interval]
...
\end{exap}
	content...
\end{comment}

\subsection{Duality between confident sets and hypothesis tests}
In this subsection, we focus on the relationship between the confident sets and hypothesis tests. Usually, we can construct a level $1-\alpha$ confident set using a deterministic size $\alpha$ test; and conversely, if we have a level $1-\alpha$ confident set, we can define a deterministic size $\alpha$ test. The correspondence is described below
\begin{enumerate}
	\item For each $\theta_0 \in \Theta$, assume there is a size $\alpha$ test for $H_0: \theta=\theta_0$:
	$$\varphi(x;\theta_0) = \begin{cases}
	1 &\quad x\notin A(\theta_0); \\ 
	0 &\quad x\in A(\theta_0).
	\end{cases}$$
	
	Recall that if $\varphi(x;\theta_0) = 1$ means $H_0$ is rejected; that is $\theta \neq \theta_0$. Thus, if the observed data $X$ is in $A(\theta_0)$, it means $\theta_0$ is closed to the real parameter $\theta$. We define 
	$$S(X) = \{ \theta\in \Theta: X\in A(\theta) \}.$$
	 
	\item Let $S(X)$ be a level $1-\alpha$ confident set. For each $\theta_0 \in \Theta$, define a test for $H_0: \theta=\theta_0$ by
	$$\varphi(x;\theta_0) = \IndOne(\theta_0 \notin S(x)).$$
\end{enumerate}  
More generally, we can construct a confident set using a randomized test. Letting $u\sim U(0,1)$ independent of $X$, set $\tilde{\varphi}_{\lambda_0}(x) = \IndOne( \varphi_{\lambda_0}(x) \geq 1-u ).$ 
\begin{prop}Let $\varphi$ be a size $\alpha$ randomized test, and $\tilde{\varphi}$ defined above.
	\begin{enumerate}[a)]
		\item $\tilde{\varphi}$ and $\varphi$ have the same power functions.
		\item $\tilde{\varphi}$ and $\varphi$ have the same size.
	\end{enumerate} 
\end{prop}
\begin{proof}
	We only consider the simplest case. Assume $\varphi = \begin{cases}
	1 \\
	\gamma \\
	0
	\end{cases}$. Then we can compute the $\EE_\theta (\tilde{\varphi})$:
	\begin{align*}
		\EE_\theta (\tilde{\varphi}) &= \PP( \varphi = 1 ) \PP( 1-\gamma > u > 0 ) + [\PP( \varphi = 1) + \PP(\varphi = \gamma)] \PP( u \leq 1-\gamma ) \\
		&= \PP(\varphi = 1) + \gamma \PP(\varphi = \gamma) \\
		&= \EE_\theta (\varphi)
	\end{align*}
	Notice they are always same whenever $\theta \in \Theta_1$ or $\in \Theta_0$.
\end{proof}

\begin{thm}Let $A: \Theta\to 2^{\underline{X}}$ and $S(X) = \{ \theta\in \Theta:  X\in A(\theta) \}$. Then
	$S(X)$ is a level $1-\alpha$ confident set if and only if $\PP_\theta(X \notin A(\theta)) \leq \alpha,\ \forall \theta \in \Theta$.
\end{thm} 

\begin{exap}
	$X_i \SimIID \mathrm{Poisson}(\lambda)$. $H_0: \lambda=\lambda_0;\ H_1:\lambda \neq \lambda_0$. Its UMPU test is of form 
	\begin{align*}
		\varphi_{\lambda_0} (x) = \begin{cases}
		1 &\quad \bar{x}<c_1,\ \bar{x}>c_2 \\
		\gamma_i &\quad \bar{x}=c_j \\
		0 &\quad \text{o.w.}
		\end{cases}
	\end{align*}
	where $c_j$ and $\gamma_j$ are chosen to have size $\alpha$. Now, we want to find a level $1-\alpha$ confident set for $\lambda$.
	
	Letting $u\sim U(0,1)$ independent of $X_i$, set 
	$$\tilde{\varphi}_{\lambda_0} = \IndOne( \varphi_{\lambda_0}(x) \geq 1-u );$$
	notice that $\tilde{\varphi}$ is a size $\alpha$ deterministic test. Its acceptance region is:
	\begin{align*}
		A(\lambda_0) = \begin{cases}
		(c_1, c_2) &\quad \min(\gamma_1,\gamma_2)> 1-u \\
		[c_1, c_2) &\quad \gamma_1 < 1-u \leq \gamma_2  \\
		(c_1, c_2] &\quad \gamma_2 < 1-u \leq \gamma_1  \\
		[c_1, c_2] &\quad \max(\gamma_1,\gamma_2) < 1 - u  \\
		\end{cases}
	\end{align*} 
\end{exap}

\begin{thm}[UMP $\implies$ UMA]
	Let $\underline{\theta}$ be a level $1-\alpha$ LCB for $\theta \in \RR$ for which
	$$\varphi(x;\theta_0) = \begin{cases}
	1 &\quad \underline{\theta}(x) > \theta_0 \\
	0 &\quad \text{o.w.}
	\end{cases}$$
	is a UMP size $\alpha$ test for $H_0: \theta=\theta_0$ vs $H_1:\theta > \theta_0$, $\forall \theta_0\in \Theta$. Then $\underline{\theta}$ is UMA.
\end{thm}

\subsection{Unbiased confident sets}
\begin{mydef}\textbf{ }
	\begin{itemize}
		\item A confident set $S(X)$ of level $1-\alpha$ is unbiased if 
		\begin{align*}
		\PP_\theta(S(X) \ni \theta) &\geq 1 -\alpha \quad \forall \theta \\
		\PP_\theta(S(X) \ni \tilde{\theta}) &\leq 1 -\alpha \quad \tilde{\theta} \neq \theta 
		\end{align*}
		
		\item A level $1-\alpha$ confident set $S(X)$ is \uline{uniformly most accurate unbiased} (UMAU) if it is unbiased and for any other unbiased level $1-\alpha$ confident set $S'(X)$
		$$\PP_\theta (S(X) \ni \tilde{\theta}) \leq \PP_\theta(S'(X) \ni \tilde{\theta}),\quad \forall \theta\in \Theta,\ \tilde{\theta} \neq \theta.$$
	\end{itemize}
\end{mydef}

\begin{thm}[UMPU $\implies$ UMPA]
	For each $\theta_0 \in \Theta$, let $A(\theta_0)$ be the acceptance region of a size $\alpha$ UMPU test of $H_0:\theta=\theta_0$ vs $H_1: \theta \neq \theta_0$. Then $S(X) = \{ \theta: X\in A(\theta) \}$ is UMAU level $1-\alpha$. 
\end{thm}

\begin{exap}
	$X_i \SimIID N(\mu,\sigma^2)$. $\mu,\sigma^2$ unknown.
\end{exap}

\subsection{Pivots}
\begin{mydef}
	Let $X \sim f_\theta$. A RV $T(X, \theta)$ is called a \uline{pivot} if its distribution is free of $\theta$. 
\end{mydef}

\begin{thm}
	If a set $C$ satisfies $\PP(T(X,\theta) \in C)\geq 1-\alpha$, then 
	$$S(X) = \{ \theta \in \Theta: T(X,\theta) \in C \}$$
	is a level $1-\alpha$ confident set.
\end{thm}
  
\begin{exap}
	$X_i \SimIID U(0,\theta)$.
\end{exap}

\subsection{Shortest length confident intervals}

\subsection{Bayes credible intervals}
\begin{mydef}
	A level $1-\alpha$ credible interval is a random set $S(X)\subset \Theta$ such that
	$$\PP(\theta \in S(X) \mid X = x) = 1 - \alpha.$$
\end{mydef}

\begin{exap}
	$X_i\SimIID \mathrm{Bin}(1,p)$. $p \sim \mathrm{Beta}(\alpha,\beta)$.
	
	Compute its posterior: $p|X=x \sim \mathrm{Beta}( \alpha+ n\bar{X}, \beta + n - n\bar{X} )$.
	
	Compute $l(x)$ and $u(x)$ such that 
	$$\PP( l(x) \leq p \leq u(x)\mid X=x ) = 1 - \alpha.$$	
	
	Then $(l(x), u(x))$ is a level $1-\alpha$ credible interval.
\end{exap}

\subsection{Large sample confident intervals} 
\begin{exap}
	$X_i\SimIID \mathrm{Bin}(1,p)$. 
	
	\begin{itemize}
		\item \textbf{Option 1}
		
		
		Notice that 
		$$\sqrt{n}(\hat{p} - p) \xrightarrow{w} N(0, p(1-p))$$
		where $\hat{p} = \bar{X}$.
	
		By Slusky's, $\sqrt{n}(\hat{p} - p)/\sqrt{ \hat{p}(1-\hat{p}) } \xrightarrow{w} N(0,1)$.
		
		$\implies \ \hat{p} \pm \sqrt{ \hat{p}(1-\hat{p})/n } z_{1-\alpha}$ is asymptotic level $1-\alpha$.
		
		\item \textbf{Option 2}
		
		Let $g:x \mapsto 2\arcsin(\sqrt{x})$. Then
		$$\sqrt{n}( g(\hat{p}) - g(p) ) \xrightarrow{w} N(0,1).$$
		
		$\implies \ g(\hat{p}) \pm  \frac{1}{\sqrt{n}} z_{1-\alpha}$ is an asymptotic level $1-\alpha$ CI for $g(p)$.
		
		$\implies \ S(X) = \{ p: | g(p) - g(\hat{p}) |\leq \frac{1}{\sqrt{n}} z_{1-\alpha} \}$ is an asymptotic level $1-\alpha$ CI for $p$.
	\end{itemize}
\end{exap}