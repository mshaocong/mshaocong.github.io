\section{Black-Scholes Model}

\subsection*{BASIC SETTING}
We consider the following processes under the objective measure $\PP$:
	\begin{align*}
	&\dd B_t = rB_t \dd t  \\ 
	& \dd S_t = \alpha S_t \dd t + \sigma S_t \dd \bar{W}_t  \tag{S}
	\end{align*}
where $r$ is a constant; $\alpha$ and $\sigma$ are $\alpha(t, S_t)$ and $\sigma(t, S_t)$ respectively.
	

\subsection*{DERIVE THE ARBITRAGE-FREE PRICE PROCESS} 
Assume $\chi = \Phi(S_T)$ is a contingent claim with the date of maturity $T$. We want to find the arbitrage-free price of $\chi = \Phi(S_T)$. 
	\begin{enumerate}
		\item Let $\Pi(t) = F(t, S_t)$ denote the price of $\chi$ at time $t$. Apply It\^o's formula:
		\begin{align*}
			\dd \Pi &= F_t \dd t + F_s \dd S + \frac{1}{2}F_{ss} \dd [S] \\
			&= F_t \dd t + F_s\underbrace{( \alpha S \dd t + \sigma S \dd \bar{W} )}_{\text{(S)}} + \frac{1}{2}F_{ss} \underbrace{\sigma^2 S^2 \dd t}_{\text{(S)}} \\
			&= \alpha_\pi \Pi  \ \dd t + \sigma_\pi \Pi \ \dd \bar{W}   
		\end{align*} 
		where $\alpha_\pi = (F_t + F_s \alpha S + \frac{1}{2}F_{ss} \sigma^2 S^2)/\Pi$ and $\sigma_\pi = (F_s \sigma S)/\Pi$.
		
		\item Form a new relative portfolio for the given two assets $S$ and $\Pi$, denoted by $(u_s, u_\pi)$. It means, if we have $V$ dollars, then we use $V\cdot u_s$ dollars to buy stocks (\textbf{Note}: totally we can buy $\frac{V_t \cdot u_s(t)}{S_t}$ stocks at time $t$) and use rest of our money to buy $\Pi$. By self-financing,
		\begin{align*}
			\dd V &= V (  u_s \frac{\dd S}{S}  + u_\pi \frac{\dd \Pi}{\Pi}  ) \\
			&= V \left[  u_s \alpha + u_\pi \alpha_\pi \right] \dd t + V \left[ u_s \sigma + u_\pi \sigma_\pi \right] \dd \bar{W}
		\end{align*}
		
		\item Notice that we can always construct a locally risk-less portfolios by choosing $(u_s, u_\pi)$ such that the equation $u_s \sigma + u_\pi \sigma_\pi = 0$ holds; so we have
		\begin{align*}
			\begin{cases}
			u_s \sigma + u_\pi \sigma_\pi = 0 \\
			u_s + u_\pi = 1
			\end{cases}
		\end{align*}
		Solve it. 
		\begin{align*}
			\begin{cases}
			u_s  =  \frac{\sigma_\pi}{\sigma_\pi - \sigma} \\
			u_\pi = \frac{-\sigma}{\sigma_\pi - \sigma} 
			\end{cases} \tag{1}
		\end{align*}
		
		\item Moreover, by the arbitrage-free condition, we must have 
		$$ u_s  \alpha +  u_\pi  \alpha_\pi = r.$$ 
		In (1), we have solved $u_s$ and $u_\pi$. And we also have solved $\alpha_\pi$ and $\sigma_\pi$. Then we obtain the \textbf{Black-Sholes equation}:
		\begin{align*}
			\begin{cases}
			F_t + rs F_s + \frac{1}{2} \sigma^2 s^2 F_{ss} - r F= 0 \\
			F(T,s) = \Phi(s)
			\end{cases} \tag{Black-Sholes}
		\end{align*}
		
		\item LAST STEP! By Feynman-Kac formula, the solution to the PDE can be represented as 
		\begin{align*}
			F(t,s) = e^{-r(T-t)}\EE^\PP_{t,s} \left[  \Phi( X_T ) \right] 
		\end{align*}  
		where $X$ is an It\^o diffusion defined by
		\begin{align*}
			 \dd X_u &= r X_u \dd u + \sigma X_u  \dd W_u, \tag{2} \\
			 X_t &= s.
		\end{align*} 
		Note that $W$ is a Brownian motion under the objective measure $\PP$.  The only difference between (2) and (S) is that the drift for $S$ is $\alpha$ rather than $r$. Applying the Girsanov's theorem, $X$ can be explained as the stock price $S$ on a new measure $\QQ$. Therefore, we re-write our basic setting under~$\QQ$ as 
		\begin{align*}
			\dd S = r S \dd t + \sigma S \dd S.
		\end{align*}
		Then we have the following useful formula:
		\begin{align*}
			F(t,s) = e^{-r(T-t)}\EE^\QQ_{t,s} \left[  \Phi( S_T ) \right] \tag{Price}
		\end{align*} 
	\end{enumerate} 

 
\begin{example}[Pricing European Call option]
	Now consider the simplest case:
	\begin{align*}
	&\dd B_t = rB_t \dd t  \\ 
	& \dd S_t = \alpha S_t \dd t + \sigma S_t \dd \bar{W}_t 
	\end{align*}
	where $r$, $\alpha$, and $\sigma$ are constant.
	
	We want to price an European call option with the underlying stock $S$, the maturity time $T$, and the strike price $K$.  Let its arbitrage-free price process bet $C(t,s)$. Let $\Phi(x) = (x- K)^+$. We directly use the formula (Price):
	\begin{align*}
		C(t,s) &= e^{-r(T-t)}\EE^\QQ_{t,s} \left[  ( S_T - K )^+ \right] \\
			&= e^{-r(T-t)} \Big\{ \underbrace{\EE^\QQ_{t,s}  \left[  S_T \IndOne\{ S_T \geq K \} \right]}_{\text{Part I}} -  K \underbrace{\EE^\QQ_{t,s} \left[  \IndOne\{ S_T \geq K \} \right]}_{\text{Part II}}  \Big\}
	\end{align*}
	Recall that $S$ is a geometric Brownian motion 
	$$S_T = S_t \exp\left\{ ( r - \frac{\sigma^2}{2} )(T-t) + \sigma(W_T - W_t) \right\}.$$
	Then we can represent Part I and Part II using the CDF of the standard normal random variable $N(\cdot)$. Finally, we find
	$$C(t,s) = e^{-r(T-t)} \left[se^{r(T-t)}N(d_1) -  K N(d_2)\right]$$
	where
	\begin{align*}
		& d_1 =  \frac{1}{ \sigma \sqrt{T-t} } \left[  \ln\frac{s}{K} + (r + \frac{1}{2}\sigma^2)(T-t)   \right], \\
		& d_2 = d_1 - \sigma \sqrt{ T - t }.
	\end{align*}
	This result is also known as \textbf{Black-Scholes formula}.
\end{example}

\begin{thm}[Put-Call parity]
	Let $P(T,K)$ and $C(T,K)$ be the price of an European put option and call option at time $0$ respectively with underlying asset $S$, maturity $T$, and strike $K$. Then
	$$P(T,K) = Ke^{-r(T-t)} + C(T,K) - S_0.$$
\end{thm}
\begin{proof}
	By (Price), we directly have
	\begin{align*}
		P(T,K) &= \EE^\QQ \left[  e^{-rT} (K - S_T)^+ \bst S_0 \right], \\
		C(T,K) &= \EE^\QQ \left[  e^{-rT} (S_T - K)^+ \bst S_0 \right].
	\end{align*}
	Then let $C(T, K) - P(T,K)$:
	\begin{align*}
		P(T,K) - C(T, K)&= e^{-rT}\EE^\QQ\left[  (K - S_T)^+ - (S_T - K)^+ \bst S_0    \right] \\
		&= e^{-rT}\EE^\QQ\left[  K - S_T \bst S_0  \right]  \\
		&= Ke^{-rT} - S_0
	\end{align*} 
\end{proof}

\begin{example}[Pricing $\ln S_T$]
	\begin{itemize}
		\item We want to find the arbitrage free price of $\Phi(S(T)) = \ln S(T)$. We directly use Theorem 7.8 in the textbook:
		\begin{align*}
		\Pi(t) = e^{-\mu(T-t)} \EE_{t,s}^Q [ \ln S(T) ]. \tag{thm7.8}
		\end{align*} 
		
		\item Now we consider the distribution of $\ln S(T)$ under the risk-neutral measure $Q$. Under the standard Black-Scholes model, 
		\begin{align*}
		d S_t = \mu S_t d t + \sigma S_t d W_t \tag{7.43-7.44}
		\end{align*}
		where $W$ is a Brownian motion under the risk-neutral measure $Q$; and we assume $S_t =s$. We know that $S$ is a geometric Brownian motion; so
		\begin{align*}
		S_T &= s \exp\left(  (\mu - \frac{\sigma^2}{2})(T-t) + \sigma(W_T-W_t)   \right) \\
		\ln S_T &= \ln s + (\mu - \frac{\sigma^2}{2})(T-t) + \sigma(W_T-W_t) \\
		\EE^Q_{t,s}\left(\ln S_T\right) &=  \ln s + (\mu - \frac{\sigma^2}{2})(T-t)
		\end{align*} 
		
		\item We plug it into (thm7.8):
		\begin{align*}
		\Pi(t) &= e^{-\mu(T-t)} \cdot \left(\ln s + (\mu - \frac{\sigma^2}{2})(T-t)\right).
		\end{align*} 
		
	\end{itemize}
\end{example}

\begin{example}[Pricing $S_T^\beta$]
	\begin{itemize}
		\item We want to find the arbitrage free price of $\Phi(S(T)) = S^\beta(T)$. We directly use Theorem 7.8 in the textbook:
		\begin{align*}
		\Pi(t) = e^{-\mu(T-t)} \EE_{t,s}^Q [ S^\beta(T) ]. \tag{thm7.8}
		\end{align*} 
		
		\item Now we consider the distribution of $S^\beta(T)$ under the risk-neutral measure $Q$. Under the standard Black-Scholes model, 
		\begin{align*}
		d S_t = \mu S_t d t + \sigma S_t d W_t \tag{7.43-7.44}
		\end{align*}
		where $W$ is a Brownian motion under the risk-neutral measure $Q$; and we assume $S_t =s$.. Using It\^o's formula to $S^\beta(t)$:
		$$d S^\beta_t = \left(\mu \beta  + \frac{\sigma^2}{2} \beta (\beta - 1)\right) S_t^\beta dt + \sigma \beta S^\beta_t dW_t.$$
		
		Obviously, $S^\beta$ is a geometric Brownian motion. We know the expectation of $S^\beta(T)$ is 
		$$\EE^Q_{t,s}(S^\beta(T)) = s^\beta \exp\left( (\mu \beta  + \frac{\sigma^2}{2} \beta (\beta - 1)) (T-t) \right).$$
		
		\item Therefore, we just plug it into the formula (thm7.8):
		$$\Pi_t = s^\beta \exp\left( (\mu (\beta-1)  + \frac{\sigma^2}{2} \beta (\beta - 1)) (T-t) \right).$$ 
	\end{itemize}
\end{example}

\begin{example}[Pricing $K \cdot\IndOne_{[\alpha,\beta]}\circ S_T$]
	\begin{itemize}
		\item First, the binary option is given by 
		$$\Phi( S_T ) = \begin{cases}
		K  \quad S_T \in [\alpha, \beta] \\
		0 \quad \text{o.w.}
		\end{cases}  $$
		
		Then we want to compute 
		\begin{align*}
		\Pi(t) = e^{-\mu(T-t)} \EE_{t,s}^Q [ \Phi(S_T) ]. \tag{thm7.8}
		\end{align*}
		
		\item Now we directly compute the expectation of $\Phi S(T)$ under the risk-neutral measure $Q$. Under the standard Black-Scholes model, we have
		\begin{align*}
		S_T = s \exp\left(  (\mu - \frac{\sigma^2}{2})(T-t) + \sigma(W_T-W_t)   \right)
		\end{align*}
		where $W$ is a Brownian motion under $Q$. Obviously, under $Q$, $W_T - W_t\sim N(0, T-t)$.
		
		Therefore, we compute its expectation 
		\begin{align*}
		\frac{1}{K}\EE_{t,s}^Q [ \Phi(S_T) ] &=  Q\left[  \alpha \leq S_T \leq \beta \right] \\
		&= Q\left[ \ln \alpha \leq  \ln s + (\mu - \frac{\sigma^2}{2})(T-t) + \sigma(W_T-W_t)  \leq \ln \beta \right] \\
		&= Q\left[ \frac{ \ln (\alpha/s) }{ \sqrt{T-t}\sigma } \leq    (\mu - \frac{\sigma^2}{2})\sqrt{T-t}/\sigma + (W_T-W_t)/\sqrt{T-t}  \leq \frac{ \ln (\beta/s) }{ \sqrt{T-t}\sigma }  \right]\\
		&= N(B) - N(A)
		\end{align*} 
		where $N$ is the cdf of $N(0,1)$, $A = \frac{ \ln (\alpha/s) }{ \sqrt{T-t}\sigma } - (\mu - \frac{\sigma^2}{2})\sqrt{T-t}/\sigma$, $B =  \frac{ \ln (\beta/s) }{ \sqrt{T-t}\sigma } - (\mu - \frac{\sigma^2}{2})\sqrt{T-t}/\sigma$.
		
		\item We plug it into (thm7.8):
		\begin{align*}
		\Pi(t) = K e^{-\mu(T-t)} \left(  N(B) - N(A) \right) 
		\end{align*}
		where $A$ and $B$ are given above.
		
	\end{itemize}
\end{example}

\begin{prop}
	Linearity
\end{prop}
 \paragraph{Note} The main idea for Exercise 9.1-9.3 is rewriting the claim as a linear combination of $\Phi_B$, $\Phi_S$ and $\Phi_{C,K}$ (given by formula (9.2)-(9.4)). Then apply Proposition 9.1 and formula (9.5)-(9.7) on page 126. 
\begin{example}

	\begin{itemize}
		\item Notice that 
		\begin{align*}
		\Phi( S_T ) &= K \IndOne \{  S_T \leq A \} + (K+A- S_T)\IndOne\{ A < S_T<K+A \}     \\
		&=  K \IndOne \{  S_T \leq A \} + (K+A -S_T) \left( \IndOne\{ S_T \geq A \} -  \IndOne\{ S_T \geq K+A  \}  \right) \\
		&= K \IndOne \{  S_T \leq A \} + K\cdot \IndOne\{ S_T \geq A \} + (A- S_T) \cdot \IndOne\{ S_T \geq A \} - (K+A -S_T) \cdot\IndOne\{ S_T \geq K+A  \}\\
		&= K \cdot \Phi_B(S_T) - \Phi_{C,A}(S_T) + \Phi_{C,K+A}(S_T)
		\end{align*} 
		where $\Phi_B(x) := 1$ and $\Phi_{C,K}(x):= (x - K) \cdot \IndOne\{ x \geq K \} = \max[ x-K,0 ].$
		\item Then by Proposition 9.1:
		\begin{align*}
		\Pi(t; \Phi) &= K\cdot \Pi(t; \Phi_B ) - \Pi(t; \Phi_{C,K}) + \Pi(t; \Phi_{C,K+A}) \\
		(\text{Using formula (9.5)-(9.7)})&= K \cdot e^{-r(T-t)} - c\left( t, S(t); K,T  \right) + c\left(  c, S(t); A+K, T \right)
		\end{align*}
		where the standard Black-Scholes model is given by
		\begin{align*}
		d S_u = r S_u d u + \sigma S_u d W_u.
		\end{align*}
	\end{itemize}
\end{example}
\begin{example}
\begin{itemize}
	\item Notice that 
	\begin{align*}
	\Phi(S_T) &= (K- S_T) \IndOne\{ S_T \leq K \} + (S_T - K) \IndOne\{ K < S_T \} \\
	&= (K- S_T)(1 - \IndOne\{ S_T > K \}) + \Phi_{C,K} \\
	&= \left(K\cdot \Phi_B - \Phi_S + 2 \Phi_{C,K}\right)(S_T)
	\end{align*}
	
	\item Then by Proposition 9.1:
	\begin{align*}
	\Pi(t; \Phi) &= K \cdot \Pi(t; \Phi_B) - \Pi(t; \Phi_S) +2 \Pi(t;  \Phi_{C,K}) \\
	&= K \cdot e^{-r(T-r)} - S(t) + 2c(t, S(t); K, T).
	\end{align*}
\end{itemize}
\end{example}
\begin{example}
	\begin{itemize}
	\item Notice that 
	\begin{align*}
	\Phi(x) &= B \IndOne\{ x > B \} +x \IndOne\{ A\leq x \leq B \} + A \IndOne\{ x<A \} \\
	&= B \IndOne\{ x > B \} +\left(  (x-A + A) \IndOne\{ x\geq A \}  - (x-B+B) \IndOne\{ x \geq B \} \right) + A \IndOne\{ x<A \} \\
	&= \Phi_{C,A}(x) - \Phi_{C,B}(x) + A\cdot \Phi_B(x)
	\end{align*} 
	
	\item Then by Proposition 9.1:
	\begin{align*}
	\Pi(t; \Phi) &= A \cdot \Pi(t; \Phi_B) + \Pi(t; \Phi_{C,A}) - \Pi(t; \Phi_{C, B}) \\
	&= A\cdot e^{-r(T-t)} + c(t,S(t); A, T) -  c(t,S(t); B, T) .
	\end{align*}
\end{itemize}
\end{example}

\subsection*{GREEKS}
Let $P(t,s)$ be the pricing function for a portfolio based on a single underlying asset. For example, it could be the price process of an European call option; it means we put all of money in this option. And we are interested in its sensitivity with respect to the price change of the underlying asset or changes in the model parameters. 
\begin{definition}[Greeks]
	\begin{align*}
	\Delta &= \frac{\partial P}{\partial s}\\
	\Gamma &= \frac{\partial^2 P}{\partial s^2}\\
	\rho &= \frac{\partial P}{\partial r}\\
	\Theta &=\frac{\partial P}{\partial t} \\
	\mathcal{V} &=  \frac{\partial P}{\partial \sigma}
	\end{align*} 
\end{definition} 

\begin{example}[Greeks for European call option]
	Let $N(\cdot)$ be the CDF and $\varphi$ be the PDF of standard normal distribution. Then the corresponding greeks are 
	\begin{align*}
		\Delta &= N(d_1)\\
		\Gamma &= \frac{\varphi(d_1)}{s\sigma \sqrt{T-t}} \\
		\rho &= K(T-t) e^{-r(T-t)} N(d_2) \\
		\Theta &= - \frac{s \varphi(d_1) \sigma}{2\sqrt{T-t}} - rK e^{-r(T-t)}N(d_2) \\
		\mathcal{V} &= s \varphi(d_1)\sqrt{T-t} 
	\end{align*}
	where 
	\begin{align*}
	& d_1 =  \frac{1}{ \sigma \sqrt{T-t} } \left[  \ln\frac{s}{K} + (r + \frac{1}{2}\sigma^2)(T-t)   \right], \\
	& d_2 = d_1 - \sigma \sqrt{ T - t }.
	\end{align*}
\end{example}

\begin{example}[Greeks for European put option]
	\begin{itemize}
		\item By Put-Call parity:
		\begin{align*}
		p(t,s) = Ke^{-r(T-t)} + c(t,s) - s \tag{9.11}
		\end{align*}
		where $c(t,s):= c(t, s; K, T)$.
		
		\item Recall that $\Delta_P:= \frac{\partial P}{\partial s}$ where $P$ is the pricing function for some options.  We take derivative w.r.t. $s$ on both sides of the equation (9.11):
		\begin{align*}
		\frac{\partial }{\partial s} p &= 0 + \frac{\partial }{\partial s} c - 1 \\
		\Delta_p &= \Delta_c - 1
		\end{align*}
		
		By Proposition 9.5, we know $\Delta_c = N(d_1)$. So 
		$$\Delta_p = N(d_1) - 1.$$
		
		\item For $\Gamma$, all steps are same. We take $\frac{\partial^2}{\partial s^2}$ on both sides of the equation (9.11):
		\begin{align*}
		\frac{\partial^2}{\partial s^2} p &= \frac{\partial^2}{\partial s^2} c \\
		\Gamma_p &= \Gamma_c
		\end{align*}
		
		By Proposition 9.5, we know $\Delta_c = \frac{\varphi(d_1)}{ s\sigma \sqrt{T-t} }$. So
		$$\Gamma_p = \frac{\varphi(d_1)}{ s\sigma \sqrt{T-t} }.$$
	\end{itemize}
\end{example}


\begin{thm}[Vega-Gamma relation]
	content...
\end{thm}

\subsection*{BARRIER OPTIONS}
In this subsection, we will consider some more complicated options; for example, the price process may be path-dependent.   

\subsection*{DIVIDENDS}
