\section{Binomial Model}
\subsection{One-Period Binomial Model}
\begin{itemize}
	\item At time $0$, we put $B_0$ dollars in the bank; and put $S_0$ dollars for stocks.
	\item At time $1$, the money for stocks will increase to $uS_0$ with probability $p$; or it also could decrease to $dS_0$ with probability $1-p$.  
\end{itemize}
\begin{figure}[H]
	\centering
	\includegraphics[width=0.4\textwidth]{binomial.png}
	\caption{One-Period Binomial Model}
\end{figure}

\subsubsection*{Portfolio optimization}
\begin{mydef} \textbf{ }
	\begin{itemize}
		\item $w_0$ - initial wealth
		\item $w_0 = a_0 S_0 + b_0 B_0$. Then $(a_0, b_0)$ is called the \textbf{portfolio/strategy}.
	\end{itemize}
\end{mydef}


\subsubsection*{Option pricing and hedging}
\begin{mydef}\textbf{ }
	\begin{itemize}
		\item \textbf{derivatives}
		\item \textbf{contingent claim}
		\item \textbf{Vanilla option}. A contract which pay $h(S_1)$ at $T=1$.
	\end{itemize} 
\end{mydef}
 
\begin{exap}[Call option]
	\textbf{Right to buy.}
	$$h(x) = (x - K)^+.$$
	If $S_1 > K$ (the stock price is larger than $K$ at time $T=1$), buy; else, never buy.
\end{exap}

\begin{exap}[Put option]
	\textbf{Right to sell.}
	$$h(x) = (K - x)^+.$$
	If $S_1 < K$ (the stock price is less than $K$ at time $T=1$), sell; else, never sell.
\end{exap}

\begin{tcolorbox}
	\begin{center}
		\textbf{Main Question}: Pricing the contract at time $0$.
	\end{center}
\end{tcolorbox}

\begin{exap}
Given the following model where $S_0=1$:
\begin{figure}[H]
	\centering
	\includegraphics[width=0.4\textwidth]{example.png}
	\caption{Example}
\end{figure}

We want to price the contract at time $t=0$. Intuitively, we consider the expectation of $S_1$:
$$\EE( S_1 - 1)^+ = 0.1 \cdot\frac{2}{3} + 0\cdot\frac{1}{3} = \frac{1}{15} \approx 0.067.$$

However, the correct answer is $0.05$. 

\end{exap}

\subsubsection*{Arbitrage - make money for free} 
	We always assume No-arbitrage (it requires $0<d<1+r<u$). Form a portfolio:
	$$V_0 = a_0 S_0 + b_0 B_0.$$ 
Then our wealth becomes:
\begin{itemize}
	\item If $S \uparrow$:
	$$V_1 = a_0 uS_0 + b_0(1+r)B_0.$$
	\item If $S \downarrow$:
	$$V_1 = a_0 dS_0 + b_0(1+r)B_0.$$
\end{itemize}

\subsubsection*{Risk-Neutral Measure}
\begin{tcolorbox} 
		\begin{thm}
			If $\exists \PP^\ast \sim \PP$ such that $\EE^\ast \frac{S_1}{1+r} = S_0$, then there is no arbitrage. 
		\end{thm} 
\end{tcolorbox}
\begin{proof}
	content...
\end{proof}

\section{Multi-Peroid Models}